\PassOptionsToPackage{unicode=true}{hyperref} % options for packages loaded elsewhere
\PassOptionsToPackage{hyphens}{url}
%
\documentclass[]{article}
\usepackage{lmodern}
\usepackage{amssymb,amsmath}
\usepackage{ifxetex,ifluatex}
\usepackage{fixltx2e} % provides \textsubscript
\ifnum 0\ifxetex 1\fi\ifluatex 1\fi=0 % if pdftex
  \usepackage[T1]{fontenc}
  \usepackage[utf8]{inputenc}
  \usepackage{textcomp} % provides euro and other symbols
\else % if luatex or xelatex
  \usepackage{unicode-math}
  \defaultfontfeatures{Ligatures=TeX,Scale=MatchLowercase}
\fi
% use upquote if available, for straight quotes in verbatim environments
\IfFileExists{upquote.sty}{\usepackage{upquote}}{}
% use microtype if available
\IfFileExists{microtype.sty}{%
\usepackage[]{microtype}
\UseMicrotypeSet[protrusion]{basicmath} % disable protrusion for tt fonts
}{}
\IfFileExists{parskip.sty}{%
\usepackage{parskip}
}{% else
\setlength{\parindent}{0pt}
\setlength{\parskip}{6pt plus 2pt minus 1pt}
}
\usepackage{hyperref}
\hypersetup{
            pdftitle={PAI Lab},
            pdfborder={0 0 0},
            breaklinks=true}
\urlstyle{same}  % don't use monospace font for urls
\usepackage[margin=1in]{geometry}
\usepackage{graphicx,grffile}
\makeatletter
\def\maxwidth{\ifdim\Gin@nat@width>\linewidth\linewidth\else\Gin@nat@width\fi}
\def\maxheight{\ifdim\Gin@nat@height>\textheight\textheight\else\Gin@nat@height\fi}
\makeatother
% Scale images if necessary, so that they will not overflow the page
% margins by default, and it is still possible to overwrite the defaults
% using explicit options in \includegraphics[width, height, ...]{}
\setkeys{Gin}{width=\maxwidth,height=\maxheight,keepaspectratio}
\setlength{\emergencystretch}{3em}  % prevent overfull lines
\providecommand{\tightlist}{%
  \setlength{\itemsep}{0pt}\setlength{\parskip}{0pt}}
\setcounter{secnumdepth}{0}
% Redefines (sub)paragraphs to behave more like sections
\ifx\paragraph\undefined\else
\let\oldparagraph\paragraph
\renewcommand{\paragraph}[1]{\oldparagraph{#1}\mbox{}}
\fi
\ifx\subparagraph\undefined\else
\let\oldsubparagraph\subparagraph
\renewcommand{\subparagraph}[1]{\oldsubparagraph{#1}\mbox{}}
\fi

% set default figure placement to htbp
\makeatletter
\def\fps@figure{htbp}
\makeatother


\title{PAI Lab}
\author{}
\date{\vspace{-2.5em}}

\begin{document}
\maketitle

At the 2023 annual conference of the International Communication
Association, I had the opportunity to moderate a panel on managing lab
research. Autumn Edwards, Grace Sun Joo Ahn, Matthew Lombard, and Tony
Liao shared their lab management and research experience, which sparked
the establishment of my own lab -- the Presence and Artificial
Intelligence (PAI) Lab.

The PAI Lab aims at conducting research on users' perceptions of,
interactions with, and evaluations of presence-evoking emerging
technologies, such as social robots, virtual agents, chatbots, virtual
reality, and augmented reality. The PAI Lab also delves into topics,
including explainable AI (XAI), Science and Technology Studies (STS),
and ubiquitous computing.

Revolving around both the science and the methods of artificial
intelligence, our lab seeks to conduct theoretically robust and
methodologically innovative research.

Drawing on an analogy with the pronunciation and significance of ``π'',
our lab symbolizes the commitment to exploring infinite and
non-recurring possibilities for emerging technology research.

We conduct research at the University of Florida, with the support of
the technologies such as
\href{https://www.aldebaran.com/en/nao}{Aldebaran Robot NAO}
\href{https://www.doublerobotics.com/}{Double Telepresence robot}
\href{https://www.meta.com/quest/quest-pro/}{Meta Quest Pro}
\href{https://www.youtube.com/watch?v=7Brb9k1zIso\&t=10s}{UBTech Robot
Alpha} \href{https://energizelab.com/}{Robot Eilik}

Below is the information of our lab group.

Fanjue Liu

Fanjue Liu is a doctoral student at the the University of Florida. Her
research investigates how people perceive and interact socially with
emerging technologies such as virtual humans. Specifically, she
investigates how people use different psychological patterns to interact
with virtual influencers, how the emergence of virtual influencers
changes the dynamics of influencer marketing, and whether virtual
influencers can be considered a viable substitute for human influencers.

Xiaobei Chen

Xiaobei Chen is a doctoral student at the College of Journalism and
Communications at the University of Florida. Her research focuses on
using emerging technologies to promote health equity and to conduct
communication skill training in healthcare.

Jiayue Li

Yunxiao Chen

Loren Ruffin

Loren Ruffin was a Professional Master's student in the CJC and
completed an M.A.~in Mass Communication. Prior to attending UF, she
received a B.S. in Communication, with a concentration in Journalism,
from East Carolina University. Her study interests included
human-machine communication, digital user experiences, and diversity,
equity, and inclusion in digital products. As part of her capstone
project, she designed and programmed a virtual reality vocabulary game
focused on accessibility specifically for deaf and hard-of-hearing
users. Post-graduation she is currently a content designer for Uber's
Safety UX team, helping to design experiences that aim to keep earners
and consumers safe on the app.

\end{document}
